\section{Determinante und Spur von Endomorphismen}

Sei $n\in \mathbb N$ und $V$ ein $K$-Vektorraum mit $\dim_K(V)=m$.

\begin{proposition}
	\proplbl{4_4_1}
	Sei $f\in \Hom_K(V,W)$, $A'$ eine Basis von $V$ und $A=M_{A'}(f)$. Sei weiter $B\in \Mat_n(K)$. Genau 
	dann gibt es eine Basis $B'$ von $V$ mit $B=M_{B'}(f)$, wenn es $S\in \GL_n(K)$ mit $B=SAS^{-1}$ gibt.
\end{proposition}
\begin{proof}
	Ist $B'$ eine Basis von $V$ mit $B=M_{B'}(f)$, so ist $B=SAS^{-1}$ mit $S=T^{A'}_{B'}$. Sei umgekehrt $B=SAS^{-1}$ mit 
	$S\in \GL_n(K)$. Es gibt eine Basis $B'$ von $V$ mit $T^{A'}_{B'}=S$, also $M_{B'}(f)=T^{A'}_{B'}\cdot M_{A'}(f)\cdot (
	T^{A'}_{B'})^{-1}=SAS^{-1}=B$: Mit $B'=(\Phi_{A'}(f_s^{-1}(e_1)),...,\Phi_{A'}(f_s^{-1}(e_n)))$ ist $\Phi_{A'}\circ f_s^{-1}=
	\id_V\circ \Phi_{B'}$, also $T^{A'}_{B'}=M_{A'}^{A'}(\id_V)=S^{-1}$. Folglich ist $T^{A'}_{B'}=(T_{A'}^{B'})^{-1}=(S^{-1})^{-1}
	=S$ nach \propref{3_6_2}.
\end{proof}

\begin{definition}[Ähnlichkeit]
	Zwei Matrizen $A,B\in \Mat_n(R)$ heißen \begriff{ähnlich}, wenn (in Zeichen $A\sim B$) es 
	$S\in \GL_n(R)$ mit $B=SAS^{-1}$ gibt.
\end{definition}

\begin{proposition}
	Ähnlichkeit von Matrizen ist eine Äquivalenzrelation auf $\Mat_n(R)$.
\end{proposition}
\begin{proof}
	\begin{itemize}
		\item Reflexivität: $A=\mathbbm{1}_n\cdot A \cdot (\mathbbm{1}_n)^{-1}$
		\item Symmetrie: $B=SAS^{-1}\Rightarrow A=S^{-1}BS=S^{-1}B(S^{-1})^{-1}$
		\item Transitivität: $B=SAS^{-1}$, $C=TBT^{-1}\Rightarrow C=TSAS^{-1}T^{-1}=(TS)A(ST)^{-1}$
	\end{itemize}
\end{proof}

\begin{proposition}
	\proplbl{4_4_4}
	Seien $A,B\in \Mat_n(R)$. Ist $A\sim B$, so ist
	\begin{align}
		\det(A)=\det(B)\notag
	\end{align}
\end{proposition}
\begin{proof}
	$B=SAS^{-1}$, $S\in \GL_n(R)$, $\det(B)=\det(S)\cdot \det(A)\cdot \det(S)^{-1}=\det(A)$ nach \propref{4_2_11} und \propref{4_2_12}
\end{proof}

\begin{definition}[Determinante eines Endomorphismus]
	Die \begriff[Endomorphismus!]{Determinante} eines Endomorphismus $f\in \End_K(V)$ ist 
	\begin{align}
		\det(f)=\det(M_B(f))\notag
	\end{align}
	wobei $B$ eine Basis von $V$ ist. (Diese ist wohldefiniert nach \propref{4_4_1} und \propref{4_4_4})
\end{definition}

\begin{proposition}
	\proplbl{4_4_6}
	Für $f,g\in \End_K(V)$ gilt:
	\begin{itemize}
		\item $\det(\id_V)=1$
		\item $\det(f\circ g)=\det(f)\cdot \det(g)$
		\item Genau dann ist $\det(f)\neq 0$, wenn $f\in \Aut_K(V)$. In diesem Fall ist $\det(f^{-1})=\det(f)^{-1}$
	\end{itemize}
\end{proposition}
\begin{proof}
	\begin{itemize}
		\item klar
		\item folgt aus \propref{3_6_6} und \propref{4_2_11}
		\item folgt aus \propref{3_6_5} und \propref{4_2_12}
	\end{itemize}
\end{proof}

\begin{definition}[Spur einer Matrix]
	Die \begriff[Matrix!]{Spur} einer Matrix $A=(a_{ij})\in \Mat_n(R)$ ist 
	\begin{align}
		\tr(A)=\sum_{i=1}^n a_{ii}\notag
	\end{align}
\end{definition}

\begin{mathematica}[Spur einer Matrix]
	Auch für die Spur einer Matrix hat Mathematica bzw. WolframAlpha eine Funktion:
	\begin{align}
		\texttt{Tr[\{\{1, 2, 3\}, \{4, 5, 6\}, \{7, 8, 9\}\}]}\notag
	\end{align}
\end{mathematica}

\begin{lemma}
	\proplbl{4_4_8}
	Seien $A,B\in \Mat_n(R)$
	\begin{itemize}
		\item $\tr: \Mat_n(R)\to R$ ist $R$-linear
		\item $\tr(A^t)=\tr(A)$
		\item $\tr(AB)=\tr(BA)$
	\end{itemize}
\end{lemma}
\begin{proof}
	in den Übungen bereits behandelt
\end{proof}

\begin{proposition}
	\proplbl{4_4_9}
	Seien $A,B\in \Mat_n(R)$. Ist $A\sim B$, so ist $\tr(A)=\tr(B)$.
\end{proposition}
\begin{proof}
	$B=SAS^{-1}$, $S\in \GL_n(R)\Rightarrow \tr(B)=\tr(SAS^{-1})\overset{\propref{4_4_8}}{=}\tr(AS^{-1}S)=\tr(A)$
\end{proof}

\begin{definition}[Spur eines Endomorphismus]
	Die \begriff[Endomorphismus!]{Spur} eines Endomorphismus $f\in \End_K(V)$ ist
	\begin{align}
		\tr(f)=\tr(M_B(f))\notag
	\end{align} 
	wobei $B$ eine Basis von $V$ ist (Diese ist wohldefiniert nach \propref{4_4_1} und \propref{4_4_9})
\end{definition}

\begin{remark}
	Im Fall $K=\mathbb R$ kann man wie in \propref{4_2_3} den Absolutbetrag der Determinante eines $f\in \End_K(K^n)$ 
	geometrisch interpretieren, nämlich als das Volumen von $f(Q)$, wobei $Q=[0,1]^n$ der Einheitsquader ist, und somit 
	als Volumenänderung durch $f$. Auch das Vorzeichen von $\det(f)$ hat eine Bedeutung: Es gibt an, ob $f$ 
	orientierungserhaltend ist. Für erste Interpretationen der Spur siehe A100.
\end{remark}